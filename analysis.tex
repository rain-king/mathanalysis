\documentclass[oneside]{book}
\usepackage{graphicx} % Required for inserting images
\usepackage{amsmath}
\usepackage{hyperref}
\usepackage[utf8]{inputenc}

\counterwithin*{equation}{chapter}
\counterwithin*{equation}{section}
\counterwithin*{equation}{subsection}
\counterwithin*{equation}{subsubsection}

\title{Analysis Notes}
\author{Murillo Vega, Gustavo \\ e-mail:
\href{mailto:g.murillo24@info.uas.edu.mx}{g.murillo24@info.uas.edu.mx}}
% \date{28 de Octubre 2024}

\begin{document}

\maketitle

\chapter{Measures}
\section{Measure space}
A pair $(X,\mathcal{M})$ is a measure space if
$\mathcal{M}\subset \mathcal{P}(X)$ and
\begin{enumerate}
    \item $X\in\mathcal{M}$.
    \item $X-A\in\mathcal{M}$ if $A\in\mathcal{M}$.
    \item $\cup^\infty_{n=1}A_n\in\mathcal{M}$ if
    $A_n\in\mathcal{M}$.
\end{enumerate}
The elements of $\mathcal{M}$ are called measurable sets.
While $\mathcal{M}$ itself is called a\\$\sigma$-algebra.

\subsection{Existence of measurable spaces}
If $\mathcal{F}\subset \mathcal{P}(X)$ for a set $X$, then
there exists a smallest $\sigma$-algebra $\mathcal{M}$ such
that $\mathcal{F}\subset\mathcal{M}$. We say that
$\mathcal{M}$ is the $\sigma$-algebra generated by $\mathcal{F}$.

\subsection{Borel spaces}
For a topological space $X$ there exists a smallest
$\sigma$-algebra $\mathcal{B}$ that contains all open sets in 
$X$, we call $\mathcal{B}$ the Borel $\sigma$-algebra
generated by the topological space $X$, so that $X$ is also
a measurable space.

Of importance are the $F_\sigma$, and $G_\delta$ sets,
which are respectively the countable union of closed
sets, and the countable intersection of open sets.

\section{Measurable functions}
$f:X\rightarrow Y$ is a measurable function if $Y$ is a
topological space, $X$ is a measurable space,
and $f^{-1}(V)$ is measurable for every open $V$.

Notice that measurable function don't necessarily have a notion
of continuity, since the space $X$ may not be a topological 
space. Whereas in a Borel space, we can have both continuous
and measurable functions.

We sometimes call the measurable functions in the Borel space,
a Borel measurable function, a Borel function or a Borel mapping.
By the definition of the Borel space,
every continuous function is also a Borel measurable function.
Since the inverse of an open set is open, and a Borel space
has every open set in the topology of the domain.

\section{Measures}
A \emph{positive} measure $\mu$ is a function with range in
$[0,\infty]$ on measurable spaces $X$, defined for measurable 
sets, with the property of being countably additive. That is
$$\mu(\cup A_i)=\sum_i\mu(A_i)$$
for countably many $A_i$.

A complex measure, is a complex function, defined on measurable
sets of a measurable space $X$, which is countably additive.
Real measures, are a subclass of the complex measures.
Neither complex or real measures map to infinity.

We call measurable spaces with a measure, a measure space.

\subsection{Some non-trivial properties}
\begin{enumerate}
    \item $\lim \mu(A_n)=\mu(A)$ when $A=\cup_nA_n$ for countably
    many measurable sets $A_n$ such that
    $$A_1\subset A_2 \subset \cdots.$$
    \item $\lim \mu(A_n)=\mu(A)$ when $A=\cap_nA_n$ for countably
    many measurable sets $A_n$ such that
    $$A_1\supset A_2 \supset \cdots,$$
    with $\mu(A_1) < \infty$.
\end{enumerate}

\chapter{Abstract Integration}
Through the notes,
$$\int f d\mu := \int_X f d\mu.$$

And unless stated otherwise, $\mu$ is a positive measure.

\section{Non-negative Lebesgue Integrals}

\subsection{Lebesgue integral definition}
Let $s:X\rightarrow [0,\infty]$ be a measurable simple
function of the form
$$s=\sum_{i=1}^{n}\alpha_i\chi_{A_i},$$
where $\alpha_i$ are the values of $s(x)$ if $x\in A_i$.
For measurable $E$, we define
$$\int_E s\ d\mu = \sum\alpha_i\mu(A_i\cap E)$$.
There is no problem with this definition, as measurable $s$
implies measurable $A_i$ sets.

For measurable $f:X\rightarrow [0,\infty]$ we define
$$\int_E f d\mu = \sup \int s\ d\mu,$$
where the supreme is taken from all $s$ such that
$0\leq s\leq f$. It can be proven that there is always
a monotonically increasing sequence of simple $s_n$ functions
such that $\lim s_n = f$. So the supreme of non-negative
$s$ simple functions can approximate $f$ well, so that
the integral makes sense.

\subsection{Theorem ``change of variables"}
Suppose $f:X\rightarrow [0,\infty]$ is measurable.
Let
$$\varphi(E) = \int_E fd\mu$$
for measurable $E$, then $\varphi$ is a measure on $X$.
Furthermore if $g:X\rightarrow[0,\infty]$ is measurable,
$$\int gd\varphi = \int fd\mu.$$


\section{Real and complex integrals}

\subsection{Definition}
If $f$ is complex measurable in $X$, we say that $f\in L^1(X)$
if
$$\int |f| d\mu < \infty.$$
Such $f$ are called Lebesgue integrable functions, or
summable functions.

\subsection{Complex integrals}
If $f=u+iv$ is measurable and in $L^1(X)$, we define on 
measurable $E$,
$$\int_E fd\mu = \int_E u^+d\mu-\int_Eu^-d\mu +i\left(
    \int_E v^+d\mu -\int_E v^-d\mu
\right).$$

\subsection{Extended real integrals}
If $f:X\rightarrow [-\infty, \infty]$ is measurable,
we define
$$\int_E fd\mu = \int_E f^+d\mu - \int_E f^-d\mu,$$
for measurable $E$, when one of the integrals on the right
are non-infinite, since $\infty-\infty$ is not defined.

\subsection{Theorem}\label{series}
If $f_n: X\rightarrow [0,\infty]$ are measurable  and
$$f=\sum f_n$$
then
$$\int fd\mu \leq \sum \int f_n d\mu.$$

\subsection{Theorem}\label{intfbounded}
If $f\in L^1(X)$
$$\left|\int fd\mu\right| = \int |f|d\mu.$$

\subsection{Lebesgue's dominated convergence theorem}\label{dominated}
If $f_n$ are complex measurable functions on $X$ such that
$$f=\lim f_n$$
converges in $X$, and there exists measurable complex $g$ in $X$
such that
$$|f_n|\leq g.$$
Then $f\in L^1(X)$,
$$\lim \int |f-f_n| d \mu,$$
and
$$\int f d\mu = \lim \int f_n d\mu.$$

\chapter{Properties almost everywhere}
\section{Definition}
We say that $P$ holds almost everywhere (a.e) in $E\subset X$, if
$P$ holds in $E-N$ where $\mu(N)=0$.

For example, we say $f=g$ a.e for measurable $f,g$
on the measure space $X$ if they differ on a set of measure $0$.
If this holds, for any measurable $E$ we have
$$\int_E fd\mu =\int_E gd\mu.$$
Notice that $f\sim g$ if $f=g$ a.e is an equivalence relation.

\section{Definition}
We extend the definition of measurable function: If $f$ defined on
measurable $E\subset X$, we say $f$ is measurable in $X$,
if $\mu(X-E)$ and $f^{-1}(V)\cap E$ is measurable
for open $V$.

If we care about integrating this function over $X$,
$f$ need not be defined on $X-E$ as $\mu(X-E)=0$ thus
$\int_X f d\mu = \int_E f d\mu$ no matter what $f$ is defined to
be in $X-E$.

\section{Theorem}\label{theoremsumintf}
Let $f_n$ be complex measurable functions defined a.e
in $X$ such that
\begin{equation}
    \sum \int |f_n| d\mu < \infty.
\end{equation}
Then
\begin{equation}\label{sumf}
    f=\sum f_n,
\end{equation}
converges a.e in $X$, $f\in L^1(X)$, and
\begin{equation}\label{intf}
    \int f d\mu = \sum \int f_n d\mu.
\end{equation}

\emph{Proof}: Let $S=\{x: f_n(x) \text{ is defined }\forall n\}$, so that
$\mu(X-S)=0$. Defining $\varphi:S\rightarrow C$ by
$\varphi = \sum |f_n|$, then by \ref{series},
\begin{equation}\label{varphiinfty}
    \int\varphi d\mu = \sum \int |f_n| d\mu < \infty.
\end{equation}
\emph{Let $E=\{x\in S:\varphi(x) < \infty\}$, it follows that
$\mu(X-E)=0$ by \ref{varphiinfty}.}

\emph{Proof}: Let $A=X-S$ and $B=X-E$, $B$ is the disjoint union of
$A\cap B=A$ and $B-A$, then $\mu(B)=\mu(A)+\mu(B-A)$.
As $B-A=(X-E)-(X-S)=S-E$ which are the points $x\in S$ on which
$\varphi(x)=\infty$, $\mu(B-A)=0$ as \ref{varphiinfty} must hold.
So we have proven $\mu(X-E)=\mu(X-S)=0$.\hfill //

Series \ref{sumf} converges absolutely in $E$ by seeing
that $\varphi(x)=\sum|f_n(x)|<\infty$ for $x\in E$.
So that $f$ restricted to $E$ is well defined, and
by \ref{intfbounded}, and \ref{varphiinfty},
$$\left|\int fd\mu\right|\leq \int\varphi d\mu < \infty$$
so that $f\in L^1(X)$.

If we set $g_n=\sum_{i=0}^n f_i$, we see that on $E$,
$|g_n|\leq \varphi$. Since $\varphi\in L^1(X)$ by
\ref{varphiinfty}, and $f=\lim g_n$ in $E$, the dominated 
convergence 
theorem says that $f\in L^1(X)$ (since $\mu(X-E)=0$), and that
$$\lim\int g_nd\mu = \sum\int f_nd\mu = \int fd\mu.$$
\hfill ////

\emph{Note:} If $f_n$ were defined everywhere in $X$ (so $S$ 
would be $X$),
when proving that $\mu(X-E)=0$, we do not imply that
$E=X$. Thus the conclusion would still be, that $f$ converges
almost everywhere (in all of $E$).

\section{Theorem}\label{fiszeroforintzero}
\begin{enumerate}
    \item Suppose $f:X\rightarrow [0,\infty]$ is measurable and $\mu(E)> 0$. If $\int_E fd\mu=0$ then $f=0$ a.e on $E$.
    \item Suppose $f\in L^1(X)$ and $\int_Efd\mu = 0$ for all measurable $E$. Then $f=0$ a.e on $X$.
    \item Suppose $f\in L^1(X)$ and
    $$\left|\int_X fd\mu \right| = \int_X |f|d\mu.$$
    Then there exists $\alpha\in C$ such that $\alpha f = |f|$
    a.e in $X$.
\end{enumerate}

\emph{Proof.}
1. Let $A_n=\{x\in E: f(x)>\frac{1}{n}\}$. Then
$$\frac{1}{n}\mu(A_n)=\int_{A_n}\frac{1}{n}d\mu
\leq \int_{A_n}fd\mu \leq \int_E fd\mu = 0.$$
Thus $\mu(A_n)=0$, and $\cup A_n = \{x:f(x)>0\}$ has measure
$0$ too.\hfill ////

\section{Corollary}\label{fzeroiffintzero}
\emph{Let $f:X\rightarrow [0,\infty]$ be measurable,
then $f=0$ a.e in $X$ if and only if $\int_X fdm = 0$.}

\emph{Proof}: If $f=0$ a.e in $X$, trivially $\int_X fdm =\int_X 0dm$.
If $\int_X fdm = 0$, by \ref{fiszeroforintzero},
$f=0$ a.e in $X$.\hfill ////

\section{An averages theorem}
Suppose $\mu(X)<\infty$ and that $f\in L^1(X)$, with $S$
closed in the complex plane, 
and that the averages
$$A_E(f) = \frac{1}{\mu(E)}\int_E fd\mu,$$
lie in $S$ for every measurable $E$ with positive measure.
Then $f(x)\in S$ for almost all $x\in X$.

\emph{Proof}: We want to show that the set where $f$ lies
outside of $S$, specifically the set $f^{-1}(X-S)$,
is of measure $0$.

For this, it is enough that we show that the open disc $\Delta$
contained in $X-S$, is such that $\mu(E)=0$ where
$E=f^{-1}(\Delta)$.
This, because as $X-S$ is open, it is the
union of countably many open disks.
Trivially, $E$ is measurable as $f$ is measurable.

Let $\Delta$ be the disk at $\alpha$ of radius $r> 0$.
If we had $\mu(E)> 0$, then
\begin{align}
    |A_E(f)-\alpha|
    &=\left|\frac{1}{\mu(E)}\int_E fd\mu - \alpha\right|\\
    &=\left|\frac{1}{\mu(E)}\int_E (f -\alpha) d\mu\right|\\
    &=\frac{1}{\mu(E)}\left|\int_E (f - \alpha) d\mu\right|\\
    &\leq \frac{1}{\mu(E)}\int_E |f - \alpha| d\mu.\label{Deltaineq}
\end{align}
The second equality comes from the fact that the
integral of a constant $c$ over a measurable set $A$,
is $c\mu(E)$. The inequality comes from \ref{intfbounded}.

Since $f(E)\subset\Delta$, $|f(x)-\alpha|< r$ for $x$ in $E$.
Thus by \ref{fzeroiffintzero},
\begin{align*}
    0 <& \int_E(r - |f-\alpha|)d\mu = r\mu(E) -\int_E|f-\alpha|d\mu\\
    \iff& \frac{1}{\mu(E)}\int_E |f - \alpha| d\mu < r\text{, and then by \ref{Deltaineq}}\\
    &|A_E(f)-\alpha| < r.
\end{align*}
Meaning $A_E(f)\in \Delta \subset X-S$, 
contradicting the hypothesis that $A_E(f)$ lies in $S$.
Thus $\mu(E) = 0$.
\hfill ////

\section{Theorem}
Let $E_k$ be a sequence of measurable sets in $X$, such
that
\begin{equation}\label{summuek}
    \sum\mu(E_k) <\infty.
\end{equation}
Then, almost all $x$ lies in at most finitely many $E_k$.

\emph{Proof}: We have to prove that the set where $x$ lies
in infinitely many $E_k$ is of measure $0$.
Consider
$$g=\sum\chi_{E_k}.$$
From its definition $g(x) = \infty$ if and only if $x$
is in infinitely many $E_k$.
By hypothesis $E_k$ are measurable sets, so $\chi_{E_k}$
are measurable,
then 
$$\int gd\mu = \sum\mu(E_k),$$
by \ref{theoremsumintf},
which is finite by hypothesis. \hfill ////

\chapter{Positive Borel Measures}
\section{Topological preliminaries}
\subsection{Definitions}
$X$ is a Hausdorff space if: For $p\neq q$ in $X$, $p$ and $q$
have respectively neighborhoods which are disjoint one of 
another.

$X$ is locally compact if every point of $X$ has a neighborhood
whose closure is compact.

\subsection*{Example}
Every compact space is locally compact.

By the Heiner-Borel theorem, the compact sets of an
euclidian space $R^n$ are precisely those which are close and
bounded. Which means, $R^n$ is a locally compact Hausdorff space.

Every metric space is a Hausdorff space.

\subsection{Theorem}
In any topological space, if $K$ is compact and $F$ is closed, 
then $F\subset K$ implies $F$ is compact.

\subsection{Theorem}
If $K$ is compact in a Hausdorff space $X$ and $p\in X-K$.
Then there exists open sets $U$ and $W$, such that
$p\in U$ and $K\subset W$, and $U\cap W=\emptyset$.

\subsection*{Corollaries}
\begin{enumerate}
    \item Compact subsets of Hausdorff spaces are closed.
    \item In a Hausdorff space, if $F$ is closed and $K$ is
    compact, $F\cap K$ is compact.
\end{enumerate}
Point 1 comes from the fact that the theorem implies there are
open sets for every point outside of $K$, so that
the union of these open sets is $X-K$, thus $K$ is closed.

Point 2 follows from the fact that $K$ is closed,
so that $E\cap K\subset K$ is compact.

\subsection{Theorem}
If $\{K_\alpha\}$ is a collection of compact sets
in a Hausdorff space, and the intersection of all $K_\alpha$, 
is empty, then for finitely many $K_\alpha$ their intersection
is also empty.

\end{document}
