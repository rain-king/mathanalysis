\documentclass[oneside]{book}
\usepackage{graphicx} % Required for inserting images
\usepackage{amsmath}
\usepackage{hyperref}
\usepackage[utf8]{inputenc}

\counterwithin*{equation}{section}
\counterwithin*{equation}{subsection}

\title{Analysis Notes}
\author{Murillo Vega, Gustavo \\ e-mail:
\href{mailto:g.murillo24@info.uas.edu.mx}{g.murillo24@info.uas.edu.mx}}
% \date{28 de Octubre 2024}

\begin{document}

\maketitle

\chapter{Abstract Integration}
Through the notes,
$$\int f d\mu := \int_X f d\mu.$$

And unless stated otherwise, $\mu$ is a positive measure.

\section{Non-negative Lebesgue Integrals}

\subsection{Lebesgue integral definition}
Let $s:X\rightarrow [0,\infty]$ be a measurable simple
function of the form
$$s=\sum_{i=1}^{n}\alpha_i\chi_{A_i},$$
where $\alpha_i$ are the values of $s(x)$ if $x\in A_i$.
For measurable $E$, we define
$$\int_E s\ d\mu = \sum\alpha_i\mu(A_i\cap E)$$.
There is no problem with this definition, as measurable $s$
implies measurable $A_i$ sets.

For measurable $f:X\rightarrow [0,\infty]$ we define
$$\int_E f d\mu = \sup \int s\ d\mu,$$
where the supreme is taken from all $s$ such that
$0\leq s\leq f$. It can be proven that there is always
a monotonically increasing sequence of simple $s_n$ functions
such that $\lim s_n = f$. So the supreme of non-negative
$s$ simple functions can approximate $f$ well, so that
the integral makes sense.

\subsection{Theorem ``change of variables"}
Suppose $f:X\rightarrow [0,\infty]$ is measurable.
Let
$$\varphi(E) = \int_E fd\mu$$
for measurable $E$, then $\varphi$ is a measure on $X$.
Furthermore if $g:X\rightarrow[0,\infty]$ is measurable,
$$\int gd\varphi = \int fd\mu.$$


\section{Real and complex integrals}

\subsection{Definition}
If $f$ is complex measurable in $X$, we say that $f\in L^1(X)$
if
$$\int |f| d\mu < \infty.$$
Such $f$ are called Lebesgue integrable functions, or
summable functions.

\subsection{Complex integrals}
If $f=u+iv$ is measurable and in $L^1(X)$, we define on 
measurable $E$,
$$\int_E fd\mu = \int_E u^+d\mu-\int_Eu^-d\mu +i\left(
    \int_E v^+d\mu -\int_E v^-d\mu
\right).$$

\subsection{Extended real integrals}
If $f:X\rightarrow [-\infty, \infty]$ is measurable,
we define
$$\int_E fd\mu = \int_E f^+d\mu - \int_E f^-d\mu,$$
for measurable $E$, when one of the integrals on the right
are non-infinite, since $\infty-\infty$ is not defined.

\subsection{Theorem}\label{series}
If $f_n: X\rightarrow [0,\infty]$ are measurable  and
$$f=\sum f_n$$
then
$$\int fd\mu = \sum \int f_n d\mu.$$

\subsection{Theorem}
If $f\in L^1(X)$
$$\left|\int fd\mu\right| = \int |f|d\mu.$$

\subsection{Lebesgue's dominated convergence theorem}\label{dominated}
If $f_n$ are complex measurable functions on $X$ such that
$$f=\lim f_n$$
converges in $X$, and there exists measurable complex $g$ in $X$
such that
$$|f_n|\leq g.$$
Then $f\in L^1(X)$,
$$\lim \int |f-f_n| d \mu,$$
and
$$\int f d\mu = \lim \int f_n d\mu.$$

\subsection{Definition}
We say that $P$ holds almost everywhere (a.e) in $E\subset X$, if
$P$ holds in $E-N$ where $\mu(N)=0$.

For example, we say $f=g$ a.e for measurable $f,g$
on the measure space $X$ if they differ on a set of measure $0$.
If this holds, for any measurable $E$ we have
$$\int_E fd\mu =\int_E gd\mu.$$
Notice that $f\sim g$ if $f=g$ a.e is an equivalence relation.

\subsection{Definition}
We extend the definition of measurable function: If $f$ defined on
measurable $E\subset X$, we say $f$ is measurable in $X$,
if $\mu(X-E)$ and $f^{-1}(V)\cap E$ is measurable
for open $V$.

If we care about integrating this function over $X$,
$f$ need not be defined on $X-E$ as $\mu(X-E)=0$ thus
$\int_X f d\mu = \int_E f d\mu$ no matter what $f$ is defined to
be in $X-E$.

\subsection{Theorem}
Let $f_n$ be complex measurable functions defined a.e
in $X$ such that
$$\sum \int |f_n| d\mu < \infty.$$
Then
$$f=\sum f_n,$$
converges a.e in $X$, $f\in L^1(X)$, and
$$\int f d\mu = \sum \int f_n d\mu.$$

PROOF: Let $S=\{x: f_n(x) \text{ is defined }\forall n\}$, so that
$\mu(X- S)=0$. Defining $\varphi:S\rightarrow C$ by
$\varphi = \sum |f_n|$, then by \ref{series},
\begin{equation}\label{varphiinfty}
    \int_S\varphi d\mu = \sum \int_S |f_n| d\mu < \infty.
\end{equation}
\emph{Let $E=\{x\in S:\varphi(x) < \infty\}$, it follows that
$\mu(X-E)=0$ by \ref{varphiinfty}.}

Proof: Let $A=X-S$ and $B=X-E$, $B$ is the disjoint union of
$A\cap B=A$ and $B-A$, then $\mu(B)=\mu(A)+\mu(B-A)$.
As $B-A=(X-E)-(X-S)=S-E$ which are the points $x\in S$ on which
$\varphi(x)=\infty$, $\mu(B-A)=0$ as \ref{varphiinfty} must hold.
So we have proven $\mu(X-E)=\mu(X-S)=0$.\hfill //


\end{document}
